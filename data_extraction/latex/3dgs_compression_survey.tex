\documentclass{article}

% Packages
\usepackage[utf8]{inputenc} % input encoding
\usepackage[T1]{fontenc} % font encoding
\usepackage{amsmath} % math enhancements
\usepackage{amsfonts} % math fonts
\usepackage{amssymb} % additional math symbols
\usepackage{graphicx} % include graphics
\usepackage{hyperref} % hyperlinks
\usepackage{cite} % improved citations
\usepackage{geometry} % page dimensions
\usepackage{authblk} % for author affiliations
\usepackage{float}
\usepackage{booktabs}
\usepackage{siunitx}
\usepackage{array}
\usepackage{varwidth} 
\usepackage[table]{xcolor}    % loads also »colortbl«
% colors for table
\definecolor{lightred}{HTML}{FF9999}
\definecolor{lightyellow}{HTML}{FFFF99}
\definecolor{lightorange}{HTML}{FFCC99}

\usepackage{makecell}
\usepackage{adjustbox}
% make text the same size even when its bold in a table
\newsavebox\CBox
\def\textBF#1{\sbox\CBox{#1}\resizebox{\wd\CBox}{\ht\CBox}{\textbf{#1}}}

% Page dimensions
\geometry{letterpaper, margin=1in}

% Title, author, and date
\title{3DGS.zip: A survey on 3D Gaussian Splatting Compression Methods}
\author[1]{Milena T. Bagdasarian}
\author[1]{Paul Knoll}
\author[1,2]{Florian Barthel}
\author[1]{Anna Hilsmann}
\author[1,2]{Peter Eisert}
\author[1]{Wieland Morgenstern}

\affil[1]{Fraunhofer Heinrich Hertz, HHI}
\affil[2]{Humboldt University of Berlin}
% make sure no date is displayed, arxiv periodically rebuilds 
% submissions which would change the date
\date{}

\begin{document}

\maketitle

\begin{abstract}

    We present a work-in-progress survey on 3D Gaussian Splatting\cite{kerbl3Dgaussians} compression methods, 
    focusing on their statistical performance across various benchmarks. This survey aims 
    to facilitate comparability by summarizing key statistics of different compression 
    approaches in a tabulated format. The datasets evaluated include TanksAndTemples\cite{TanksAndTemples}, 
    MipNeRF360\cite{MipNeRF360}, DeepBlending\cite{DeepBlending}, and SyntheticNeRF\cite{SyntheticNeRF}. For each method, we report the Peak 
    Signal-to-Noise Ratio (PSNR), Structural Similarity Index (SSIM), Learned Perceptual 
    Image Patch Similarity (LPIPS), and the resultant size in megabytes (MB), as 
    provided by the respective authors.
    This is an ongoing, open project, and we invite contributions from the research community 
    as GitHub issues or pull requests. Please visit \mbox{\bfseries\url{http://w-m.github.io/3dgs-compression-survey/}}
    for more information and a sortable version of the table.
    
\end{abstract}

% \section{Introduction}
% % Novel view synthesis, nerf, 3DGS ..(vs. traditional photogrammetry -> not scalable)
% % objective: comprehensive overview of the current state of 3DGS compression methods
% % why is compression important in 3DGS? 
% % applications in areas such as computer graphics, virtual reality (VR), augmented reality (AR), and real-time rendering.

\section{Scope of this survey}
% table; comparison of statistics of compression methods
% types of compression
% reproducability

In this survey, we focus on compression methods for 3D Gaussian Splatting (3DGS), aiming to optimize memory usage while preserving visual quality and real-time rendering speed. We provide a comprehensive comparison of various compression techniques, with quantitative results for the most commenly used datasets summarized in a tabulated format. Our goal is to ensure transparency and reproducibility of the included approaches. Additionally, we offer a brief explanation of each pipeline and discuss main compression approaches. Rather than covering all existing 3DGS methods, our focus is specifically on their compression techniques; for a broader overview of 3DGS methods and applications, we refer readers to \cite{wu2024recent,fei20243d}. While we include many common approaches shared between neural radiance field (NeRF)\cite{mildenhall2020nerf} compression and 3DGS compression, we direct readers to \cite{li2023compressing,chen2024far} for NeRF-specific compression methods.

\rotatebox{90}{
    \begin{minipage}{\textheight} % Adjust the minipage width to fit the content
        \section*{Survey Table}
        % Have a look into ../tex_templates/preamble.tex for packages and color definitions 

        % color every second line in grey
        \rowcolors{2}{gray!25}{white}

        \footnotesize
        \setlength{\tabcolsep}{3pt}
        
        % center the table
        \begin{adjustbox}{center}
           \begin{tabular}{ll|llllll|llllll|llllll|llllll}
\toprule
Method & Rank & \multicolumn{6}{c|}{TanksAndTemples} & \multicolumn{6}{c|}{MipNeRF360} & \multicolumn{6}{c|}{DeepBlending} & \multicolumn{6}{c}{SyntheticNeRF} \\
 & \tiny  & \tiny PSNR$\uparrow$ & \tiny SSIM$\uparrow$ & \tiny LPIPS$\downarrow$ & \tiny \makecell{Size \\ MB$\downarrow$} & \tiny k Gauss & \tiny b/G & \tiny PSNR$\uparrow$ & \tiny SSIM$\uparrow$ & \tiny LPIPS$\downarrow$ & \tiny \makecell{Size \\ MB$\downarrow$} & \tiny k Gauss & \tiny b/G & \tiny PSNR$\uparrow$ & \tiny SSIM$\uparrow$ & \tiny LPIPS$\downarrow$ & \tiny \makecell{Size \\ MB$\downarrow$} & \tiny k Gauss & \tiny b/G & \tiny PSNR$\uparrow$ & \tiny SSIM$\uparrow$ & \tiny LPIPS$\downarrow$ & \tiny \makecell{Size \\ MB$\downarrow$} & \tiny k Gauss & \tiny b/G \\
\midrule
HAC-lowrate & \cellcolor{lightred}3.9 & \cellcolor{lightyellow}24.04 & 0.846 & 0.187 & \cellcolor{lightred}8.1 &  &  & \cellcolor{lightyellow}27.53 & 0.807 & 0.238 & \cellcolor{lightorange}15.3 &  &  & 29.98 & 0.902 & 0.269 & \cellcolor{lightred}4.3 &  &  & 33.24 & 0.967 & 0.037 & \cellcolor{lightred}1.2 &  &  \\
HAC-highrate & \cellcolor{lightorange}4.0 & \cellcolor{lightred}24.40 & \cellcolor{lightorange}0.853 & \cellcolor{lightorange}0.177 & 11.2 &  &  & \cellcolor{lightred}27.77 & \cellcolor{lightorange}0.811 & \cellcolor{lightyellow}0.230 & 21.9 &  &  & \cellcolor{lightyellow}30.34 & 0.906 & 0.258 & \cellcolor{lightyellow}6.3 &  &  & \cellcolor{lightorange}33.71 & \cellcolor{lightyellow}0.968 & \cellcolor{lightyellow}0.034 & \cellcolor{lightyellow}1.9 &  &  \\
gsplat-1.00M & \cellcolor{lightyellow}4.5 & 24.03 & \cellcolor{lightred}0.857 & \cellcolor{lightred}0.163 & 15.4 & 1,000 & \cellcolor{lightyellow}129 & 27.29 & \cellcolor{lightorange}0.811 & \cellcolor{lightorange}0.229 & \cellcolor{lightorange}15.3 & \cellcolor{lightorange}1,000 & \cellcolor{lightyellow}128 &  &  &  &  &  &  &  &  &  &  &  &  \\
IGS-Low & 5.2 & 23.70 & 0.836 & 0.227 & \cellcolor{lightorange}8.4 &  &  & 27.33 & 0.809 & 0.257 & \cellcolor{lightred}12.5 &  &  & \cellcolor{lightorange}30.63 & 0.904 & 0.293 & \cellcolor{lightyellow}6.3 &  &  & \cellcolor{lightyellow}33.36 & \cellcolor{lightorange}0.971 & 0.036 & \cellcolor{lightorange}1.8 &  &  \\
IGS-High & 5.5 & \cellcolor{lightorange}24.05 & \cellcolor{lightyellow}0.849 & 0.210 & 12.5 &  &  & \cellcolor{lightorange}27.62 & \cellcolor{lightred}0.819 & 0.247 & 25.4 &  &  & \cellcolor{lightred}32.33 & \cellcolor{lightred}0.924 & 0.253 & 7.7 &  &  & \cellcolor{lightred}34.18 & \cellcolor{lightred}0.975 & \cellcolor{lightred}0.032 & 2.7 &  &  \\
Navaneet et al. 32K & 7.2 & 23.44 & 0.838 & 0.198 & 13.0 & \cellcolor{lightred}520 & 210 & 27.12 & 0.806 & 0.240 & 19.0 & \cellcolor{lightred}845 & 189 & 29.90 & \cellcolor{lightyellow}0.907 & \cellcolor{lightyellow}0.251 & 13.0 & \cellcolor{lightred}554 & 197 &  &  &  &  &  &  \\
Navaneet et al. 16K & 7.7 & 23.39 & 0.836 & 0.200 & 12.0 &  &  & 27.03 & 0.804 & 0.243 & 18.0 &  &  & 29.90 & 0.906 & 0.252 & 12.0 &  &  &  &  &  &  &  &  \\
RDO-Gaussian & 8.0 & 23.34 & 0.835 & 0.195 & 11.5 & 907 & \cellcolor{lightorange}106 & 27.05 & 0.802 & 0.239 & 22.4 & 1,859 & \cellcolor{lightorange}101 & 29.63 & 0.902 & 0.252 & 17.2 & 1,474 & \cellcolor{lightorange}98 & 33.12 & 0.967 & 0.035 & 2.2 & \cellcolor{lightred}132 & \cellcolor{lightorange}139 \\
Reduced3DGS & 8.2 & 23.57 & 0.840 & 0.188 & 14.0 & \cellcolor{lightyellow}680 & 173 & 27.10 & 0.809 & \cellcolor{lightred}0.226 & 29.0 & 1,460 & 167 & 29.63 & 0.902 & \cellcolor{lightred}0.249 & 18.0 & 1,010 & \cellcolor{lightyellow}149 &  &  &  &  &  &  \\
Morgenstern et al. w/o SH & 8.5 & 23.15 & 0.828 & 0.198 & \cellcolor{lightyellow}8.9 & 1,207 & \cellcolor{lightred}62 & 26.56 & 0.791 & 0.241 & \cellcolor{lightyellow}15.9 & 2,149 & \cellcolor{lightred}62 & 29.12 & 0.892 & 0.270 & \cellcolor{lightorange}5.4 & \cellcolor{lightorange}800 & \cellcolor{lightred}57 & 31.37 & 0.959 & 0.043 & \cellcolor{lightyellow}1.9 & \cellcolor{lightyellow}175 & \cellcolor{lightred}90 \\
Morgenstern et al. & 10.2 & 23.56 & 0.837 & \cellcolor{lightyellow}0.186 & 21.7 & 1,242 & 147 & 27.08 & 0.799 & \cellcolor{lightyellow}0.230 & 38.4 & 2,176 & 148 & 29.26 & 0.894 & 0.268 & 16.9 & \cellcolor{lightyellow}890 & 159 & 33.23 & 0.966 & \cellcolor{lightyellow}0.034 & 3.9 & \cellcolor{lightorange}157 & \cellcolor{lightyellow}210 \\
Compressed3D & 10.2 & 23.32 & 0.832 & 0.194 & 17.3 &  &  & 26.98 & 0.801 & 0.238 & 28.8 &  &  & 29.38 & 0.898 & 0.253 & 25.3 &  &  & 32.94 & 0.967 & \cellcolor{lightorange}0.033 & 3.7 &  &  \\
Compact3DGS+PP & 11.1 & 23.32 & 0.831 & 0.202 & 20.9 &  &  & 27.03 & 0.797 & 0.247 & 29.1 &  &  & 29.73 & 0.900 & 0.258 & 23.8 &  &  & 32.88 & \cellcolor{lightyellow}0.968 & \cellcolor{lightyellow}0.034 & 2.7 &  &  \\
EAGLES & 11.5 & 23.37 & 0.84 & 0.20 & 29.0 &  &  & 27.23 & \cellcolor{lightyellow}0.81 & 0.24 & 54.0 &  &  & 29.86 & \cellcolor{lightorange}0.91 & \cellcolor{lightorange}0.25 & 52.0 &  &  &  &  &  &  &  &  \\
Scaffold-GS & 11.9 & 23.96 & \cellcolor{lightorange}0.853 & \cellcolor{lightorange}0.177 & 87.0 &  &  & 27.50 & 0.806 & 0.252 & 253.9 &  &  & 30.21 & 0.906 & 0.254 & 66.0 &  &  &  &  &  &  &  &  \\
Compact3DGS & 12.4 & 23.32 & 0.831 & 0.201 & 39.4 & 836 & 395 & 27.08 & 0.798 & 0.247 & 48.8 & 1,388 & 295 & 29.79 & 0.901 & 0.258 & 43.2 & 1,058 & 342 & 33.33 & \cellcolor{lightyellow}0.968 & \cellcolor{lightyellow}0.034 & 5.5 &  &  \\
LightGaussian & 12.8 & 23.11 & 0.817 & 0.231 & 22.0 &  &  & 27.28 & 0.805 & 0.243 & 42.0 &  &  &  &  &  &  &  &  & 32.72 & 0.965 & 0.037 & 7.8 &  &  \\
EAGLES-Small & 13.3 & 23.10 & 0.82 & 0.22 & 19.0 & \cellcolor{lightorange}650 & 245 & 26.94 & 0.80 & 0.25 & 47.0 & \cellcolor{lightyellow}1,330 & 296 & 29.92 & 0.90 & \cellcolor{lightorange}0.25 & 33.0 & 1,190 & 233 &  &  &  &  &  &  \\
\bottomrule
\end{tabular}
        \end{adjustbox}
        \newline\newline
        \noindent Note: The best methods in each category are highlighted (\colorbox{lightred}{fist}, \colorbox{lightorange}{second}, \colorbox{lightyellow}{third}). The ranks represent the average rankings of the methods across all available datasets. The quality metrics PSNR, SSIM, and LPIPS are equally weighted with the model size, meaning they each contribute one-sixth to the ranks, while the size contributes half.

    \end{minipage}
}
\section{Datasets and Evaluation Statistics}

\subsection{Datasets}
% % describe all datasets briefly
Performance and quality assessment of 3D Gaussian Splatting algorithms is typically performed on multiple datasets. These datasets provide 3D scenes or objects with various properties, such as varying levels of detail, lighting conditions, and complexities, which allow for comprehensive evaluation of the algorithms. \\
In our survey we include Tanks and Temples\cite{TanksAndTemples}, MipNerf360\cite{MipNeRF360}, Deep Blending\cite{DeepBlending} as real-world datasets, and Synthetic NeRF\cite{SyntheticNeRF} as a synthetic dataset. From Tanks and Temples we include ``truck'' and ``train'' two unbounded outdoor scenes wich have a centered view point. The MipNerf360 dataset also has a centered view point but includes in- and outdoor scenes. The following scenes are included:  ``bicycle'', ``bonsai'', ``counter'', ``flowers'', ``garden'', ``kitchen'', ``room'', ``stump'', ``treehill''. From the Deep Blending dataset we include ``Dr Johnson'' and ``Palyroom'' two indoor scenes with a viewpoint directed outward. The synthetic scenes: chair, drums, ficus, hotdog, lego, material, mic, ship stem from the SyntheticNeRF dataset.

\subsection{Evaluation Statistics}
% !!! check gsplat-protocol / check downsampling factor in 3DGS; maybe in code. 
% % describe eval statistics used in the table
% \section{Datasets and Evaluation Statistics}
%     % describe all datasets briefly
%     % describe eval statistics used in the table

% \section{Fundamentals of 3D Gaussian Splatting and Compression}

%     \subsection{3D Gaussian Splatting}

%         3D Gaussian Splatting (3DGS)\cite{kerbl3Dgaussians} is a 3D scene representation based on rasterization used to perform novel view synthesis. While more traditonal methods rely on polygonal meshes or voxel grids, 3D Gaussian Splatting relys on a set of overlapping Gaussian functions (or "splats") to model the appearence of surfaces or volumes in 3D space. A "splat" in 3DGS is a 3D Gaussian ditribution that is described by its position (XYZ), Covariance (stretch and scale), color (RGB) and Alpha (transparency). 
%         % 3D Gaussian Splatting; basic principles; usage in 3D graphics and rendering.
%         % Data representation in 3DGS; Gaussian splats; challanges such as data size and rendering performance
%         % explain need for compression; storage efficiency; computation efficiency

%     \section{Compression in the context of 3D Gaussian Splatting}
%         % What can be compressed? What do we call compression?
%         % spatial data, color and intensity, (temporal data - not focused on)
%         % densification and pruning 

% \section{Classification of Compression Methods}
%     % Vector quantization
%     % Anchor-based approaches
%     % Pruning 
%     % ... ???

\section{Description of Included Compression Approaches}
    \input{3dgs_contributions.tex}
    % contribution summaries from table 
    % possibly use color encoding

% \section{Discussion}

% \section{Future Directions}

% \section{Conclusion}


% Bibliography (if needed)
\bibliographystyle{habbrv}
\bibliography{../../methods_compression, ../../datasets, survey} % assuming you have a references.bib file

\end{document}