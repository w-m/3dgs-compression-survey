\documentclass{article}

% Packages
\usepackage[utf8]{inputenc} % input encoding
\usepackage[T1]{fontenc} % font encoding
\usepackage{amsmath} % math enhancements
\usepackage{amsfonts} % math fonts
\usepackage{amssymb} % additional math symbols
\usepackage{graphicx} % include graphics
\usepackage{hyperref} % hyperlinks
\usepackage{cite} % improved citations
\usepackage{geometry} % page dimensions
\usepackage{authblk} % for author affiliations
\usepackage{float}
\usepackage{booktabs}
\usepackage{siunitx}
\usepackage{array}
\usepackage{varwidth} 
\usepackage[table]{xcolor}    % loads also »colortbl«
% colors for table
\definecolor{lightred}{HTML}{FF9999}
\definecolor{lightyellow}{HTML}{FFFF99}
\definecolor{lightorange}{HTML}{FFCC99}

\usepackage{makecell}
\usepackage{adjustbox}
% make text the same size even when its bold in a table
\newsavebox\CBox
\def\textBF#1{\sbox\CBox{#1}\resizebox{\wd\CBox}{\ht\CBox}{\textbf{#1}}}

% Page dimensions
\geometry{letterpaper, margin=1in}

% Title, author, and date
\title{3DGS.zip: A survey on 3D Gaussian Splatting Compression Methods}
\author[1]{Milena T. Bagdasarian}
\author[1]{Paul Knoll}
\author[1,2]{Florian Barthel}
\author[1]{Anna Hilsmann}
\author[1,2]{Peter Eisert}
\author[1]{Wieland Morgenstern}

\affil[1]{Fraunhofer Heinrich Hertz, HHI}
\affil[2]{Humboldt University of Berlin}
% make sure no date is displayed, arxiv periodically rebuilds 
% submissions which would change the date
\date{}

\begin{document}

\maketitle

\begin{abstract}

    We present a work-in-progress survey on 3D Gaussian Splatting\cite{kerbl3Dgaussians} compression methods, 
    focusing on their statistical performance across various benchmarks. This survey aims 
    to facilitate comparability by summarizing key statistics of different compression 
    approaches in a tabulated format. The datasets evaluated include TanksAndTemples\cite{TanksAndTemples}, 
    MipNeRF360\cite{MipNeRF360}, DeepBlending\cite{DeepBlending}, and SyntheticNeRF\cite{SyntheticNeRF}. For each method, we report the Peak 
    Signal-to-Noise Ratio (PSNR), Structural Similarity Index (SSIM), Learned Perceptual 
    Image Patch Similarity (LPIPS), and the resultant size in megabytes (MB), as 
    provided by the respective authors.
    This is an ongoing, open project, and we invite contributions from the research community 
    as GitHub issues or pull requests. Please visit \mbox{\bfseries\url{http://w-m.github.io/3dgs-compression-survey/}}
    for more information and a sortable version of the table.
    
\end{abstract}

% \section{Introduction}
% % Novel view synthesis, nerf, 3DGS ..(vs. traditional photogrammetry -> not scalable)
% % objective: comprehensive overview of the current state of 3DGS compression methods
% % why is compression important in 3DGS? 
% % applications in areas such as computer graphics, virtual reality (VR), augmented reality (AR), and real-time rendering.

\section*{Scope of this survey}
% table; comparison of statistics of compression methods
% types of compression
% reproducability

In this survey, we focus on compression methods for 3D Gaussian Splatting (3DGS), aiming to optimize memory usage while preserving visual quality and real-time rendering speed. We provide a comprehensive comparison of various compression techniques, with quantitative results for the most commenly used datasets summarized in a tabulated format. Our goal is to ensure transparency and reproducibility of the included approaches. Additionally, we offer a brief explanation of each pipeline and discuss main compression approaches. Rather than covering all existing 3DGS methods, our focus is specifically on their compression techniques; for a broader overview of 3DGS methods and applications, we refer readers to \cite{wu2024recent,fei20243d}. While we include many common approaches shared between neural radiance field (NeRF)\cite{mildenhall2020nerf} compression and 3DGS compression, we direct readers to \cite{li2023compressing,chen2024far} for NeRF-specific compression methods.

% \section{Fundamentals of 3D Gaussian Splatting and Compression}
% \subsection{3D Gaussian Splatting}

% 3D Gaussian Splatting (3DGS)\cite{kerbl3Dgaussians} is a 3D scene representation based on rasterization used to perform novel view synthesis. While more traditonal methods rely on polygonal meshes or voxel grids, 3D Gaussian Splatting relys on a set of overlapping Gaussian functions (or "splats") to model the appearence of surfaces or volumes in 3D space. A "splat" in 3DGS is a 3D Gaussian ditribution that is described by its position (XYZ), Covariance (stretch and scale), color (RGB) and Alpha (transparency). 
% % 3D Gaussian Splatting; basic principles; usage in 3D graphics and rendering.
% % Data representation in 3DGS; Gaussian splats; challanges such as data size and rendering performance
% % explain need for compression; storage efficiency; computation efficiency

% \section{Compression in the context of 3D Gaussian Splatting}
% \subsection{}

% \section{Datasets and evaluation statistics}
% % TODO describe the required datasets and statistics

% \section{Discussion}