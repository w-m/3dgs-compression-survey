% \section{Fundamentals of 3D Gaussian Splatting and Compression}
%     \subsection{3D Gaussian Splatting}

%         3D Gaussian Splatting (3DGS)\cite{kerbl3Dgaussians} is a 3D scene representation based on rasterization used to perform novel view synthesis. While more traditonal methods rely on polygonal meshes or voxel grids, 3D Gaussian Splatting relys on a set of overlapping Gaussian functions (or "splats") to model the appearence of surfaces or volumes in 3D space. A "splat" in 3DGS is a 3D Gaussian ditribution that is described by its position (XYZ), Covariance (stretch and scale), color (RGB) and Alpha (transparency). 
%         % 3D Gaussian Splatting; basic principles; usage in 3D graphics and rendering.
%         % Data representation in 3DGS; Gaussian splats; challanges such as data size and rendering performance
%         % explain need for compression; storage efficiency; computation efficiency

%     \section{Compression in the context of 3D Gaussian Splatting}
%         % What can be compressed? What do we call compression?
%         % spatial data, color and intensity, (temporal data - not focused on)
%         % densification and pruning 

% \section{Classification of Compression Methods}
%     % Vector quantization
%     % Anchor-based approaches
%     % Pruning 
%     % ... ???

\section{Description of Included Compression Approaches}
    \input{3dgs_contributions.tex}
    % contribution summaries from table 
    % possibly use color encoding

% \section{Discussion}

% \section{Future Directions}

% \section{Conclusion}
